% Orginaltemplate von:
%
% Bestätigung über Sachzuwendungen, 2014
% Uwe Ziegenhagen, ziegenhagen@gmail.com
% http://code.google.com/p/spendenquittungen-mit-latex/
%
% geändert und angepasst von Frank Löffler <frank.loeffler@uni-jena.de>

\documentclass[12pt,ngerman]{scrartcl}
\usepackage[utf8]{inputenc}
\usepackage[T1]{fontenc}
\usepackage{booktabs}
\usepackage{babel}
\usepackage{graphicx}
\usepackage{csquotes}
\usepackage{paralist}
\usepackage{dashrule}
\usepackage{mdframed}
\usepackage{wasysym}
\usepackage{tabu}
\usepackage{ifthen}
\usepackage{hyperref}

\newboolean{sammel} % Sammelbestätigung
\setboolean{sammel}{false}

\renewcommand{\familydefault}{\sfdefault}
\usepackage[scaled=0.7]{helvet}
\pagestyle{empty}

\newcommand{\marked}{\scalebox{1.5}{\XBox}~} % \CheckedBox
\newcommand{\notmarked}{\scalebox{1.5}{\Square}}
\usepackage[]{eurosym}

\usepackage{xcolor}
\usepackage[a4paper,left=2cm,right=2cm,top=1cm,bottom=1cm]{geometry}
\setlength{\parindent}{0pt}
\setlength{\parskip}{0pt}


\mdfdefinestyle{MyFormStyle}{%
    linewidth=1pt,
    skipbelow=\topskip,
    skipabove=\topskip
}

\newcommand{\MyForm}[2][1.0cm]{% 
    \begin{mdframed}[style=MyFormStyle]% 
    {\noindent\footnotesize#2}\\% 
    \TextField[format={var f =f.textFont = 'Verdana';f.strokeColor     =['T'];f.fillColor=['T']}, width=\linewidth, height=#1, charsize=10pt]{ }% 
    \end{mdframed}% 
}

\newcommand{\MyFormBox}[3][1.0cm]{%
    \begin{mdframed}[style=MyFormStyle]%
    {\noindent\footnotesize#2 \vspace*{1em} \par\normalsize \TextField[format={var f =f.textFont = 'Verdana';f.strokeColor     =['T'];f.fillColor=['T']}, multiline=true, width=\linewidth, height=#1, charsize=10pt, default=#3]{\mbox{}}}%
    \end{mdframed}%
}

\begin{document}
\begin{Form}
\MyFormBox[1.0cm]{Aussteller (Bezeichnung und Anschrift der steuerbegünstigten Einrichtung)}{de-RSE e.V.}

{\bfseries\large \ifthenelse{\boolean{sammel}}{Sammelbestätigung}{Bestätigung} über Geldzuwendungen/Mitgliedsbeiträge}\vspace*{1em}

{\footnotesize im Sinne des § 10b des Einkommensteuergesetzes an eine der in § 5 Abs. 1 Nr. 9 des Körperschaftsteuergesetzes bezeichneten Körperschaften, Personenvereinigungen oder Vermögensmassen}

\MyFormBox[2.0cm]{Name und Anschrift des Zuwendenden}{}

\begin{tabu}{|[1pt]p{0.2\textwidth}|[1pt]p{0.52\textwidth} |[1pt]p{0.2\textwidth}|[1pt]} \tabucline[1pt]{-}
\scriptsize \ifthenelse{\boolean{sammel}}{Summe}{Betrag} der Zuwendungen - in Ziffern - & \scriptsize- in Buchstaben - & \scriptsize \ifthenelse{\boolean{sammel}}{Zeitraum der Sammelbestätigung}{Tag der Zuwendung} \\ 
\vspace*{1em} & & \\ 
\hspace*{7em} \euro &  &  \\
\vspace*{1em} & & \\ \tabucline[1pt]{-}
\end{tabu}

\ifthenelse{\boolean{sammel}}{}{\vspace*{1em}Es handelt sich um den Verzicht auf Erstattung von Aufwendungen: 	\hspace{1em}	Ja  \notmarked	\hspace{1em}	Nein  \notmarked} \vspace*{1em}


\notmarked Wir sind wegen Förderung (Angabe des begünstigten Zwecks / der begünstigten Zwecke) \hdashrule{2cm}{1pt}{1pt}  nach dem letzten uns zugegangenen Freistellungsbescheid bzw. nach der Anlage zum Körperschaftssteuerbescheid des Finanzamts \hdashrule{2cm}{1pt}{1pt} StNr \hdashrule{2cm}{1pt}{1pt} vom \hdashrule{2cm}{1pt}{1pt} für den letzten Veranlagungszeitraum \hdashrule{2cm}{1pt}{1pt} nach §~5 Abs. 1 Nr.~9 des Körperschaftssteuergesetzes von der Körperschaftssteuer und nach §~3 Nr.~6 des Gewerbesteuergesetzes von der Gewerbesteuer befreit. \vspace{2em}
  
\notmarked Die Einhaltung der satzungsmäßigen Voraussetzungen nach den §§ 51, 59, 60 und 61 AO wurde vom Finanzamt \hdashrule{2cm}{1pt}{1pt} StNr. \hdashrule{2cm}{1pt}{1pt} mit Bescheid vom\hdashrule{2cm}{1pt}{1pt} nach § 60 AO gesondert festgestellt. Wir fördern nach unserer Satzung (Angabe des begünstigten Zwecks / der begünstigten Zwecke) \hdashrule{2cm}{1pt}{1pt}
 
\begin{mdframed}[style=MyFormStyle]%
Es wird bestätigt, dass die Zuwendung nur zur Förderung der begünstigten Zwecke 1, 2, 3 und 4 AO verwendet wird. 
\end{mdframed} 

\ifthenelse{\boolean{sammel}}{\vspace*{0.5em}Es wird bestätigt, dass über die in der Gesamtsumme enthaltenen Zuwendungen keine weiteren Bestätigungen, weder formelle Zuwendungsbestätigungen noch Beitragsquittungen o.ä., ausgestellt wurden und werden. 

\vspace*{0.5em}Ob es sich um den Verzicht auf Erstattung von Aufwendungen handelt, ist der Anlage zur Sammelbestätigung zu entnehmen.}{}

\vspace*{2.5em} 

Berlin, den  \hspace*{24em} 

\hrule

\vspace*{0.5em} (Ort, Datum und Unterschrift des Zuwendungsempfängers) 

\paragraph{Hinweis:} Wer vorsätzlich oder grob fahrlässig eine unrichtige Zuwendungsbestätigung erstellt oder wer veranlasst, dass 
Zuwendungen nicht zu den in der Zuwendungsbestätigung angegebenen steuerbegünstigten Zwecken verwendet 
werden, haftet für die Steuer, die dem Fiskus durch einen etwaigen Abzug der Zuwendungen beim Zuwendenden entgeht (§ 10b Abs. 4 EStG, § 9 Abs. 3 KStG, § 9 Nr. 5 GewStG). 

Diese Bestätigung wird nicht als Nachweis für die steuerliche Berücksichtigung der Zuwendung anerkannt, wenn das Datum des Freistellungsbescheides länger als 5 Jahre bzw. das Datum der der Feststellung der Einhaltung der satzungsmäßigen Voraussetzungen nach § 60 Abs. 1 AO länger als 3 Jahre seit Ausstellung des Bescheides zurückliegt (§63 Abs. 5 AO). 

\ifthenelse{\boolean{sammel}}{
\clearpage

{\bfseries\large Anlage zur Sammelbestätigung} \vspace*{2em}

\begin{tabular}{p{0.2\textwidth}p{0.2\textwidth}p{0.3\textwidth}r} \toprule
\bfseries\footnotesize Datum der Zuwendung & \bfseries\footnotesize Art der Zuwendung & \bfseries\footnotesize Verzicht auf die Erstattung von Aufwendungen (ja/nein) & \bfseries\footnotesize Betrag \\ \midrule
01.01.2013 & Mitgliedsbeitrag & nein & 123,00 \euro \\ \midrule[1pt]
Summe: & & & 123,00 \euro \\ \bottomrule[1pt]\bottomrule[1pt]
\end{tabular}}{}


\end{Form}
\end{document}

